\documentclass{article}

\usepackage{fancyhdr}
\usepackage{extramarks}
\usepackage{amsmath}
\usepackage{amsthm}
\usepackage{amsfonts}
\usepackage{tikz}
\usepackage[plain]{algorithm}
\usepackage{algpseudocode}

\usetikzlibrary{automata,positioning}

%
% Basic Document Settings
%

\topmargin=-0.45in
\evensidemargin=0in
\oddsidemargin=0in
\textwidth=6.5in
\textheight=9.0in
\headsep=0.25in

\linespread{1.1}

\pagestyle{fancy}
\lhead{\hmwkAuthorName}
\chead{\hmwkClass\ : \hmwkTitle}
\rhead{\firstxmark}
\lfoot{\lastxmark}
\cfoot{\thepage}

\renewcommand\headrulewidth{0.4pt}
\renewcommand\footrulewidth{0.4pt}

\setlength\parindent{0pt}

%
% Create Problem Sections
%

\newcommand{\enterProblemHeader}[1]{
    \nobreak\extramarks{}{Problème \arabic{#1} continué sur la prochaine page\ldots}\nobreak{}
    \nobreak\extramarks{Problème \arabic{#1} (continué)}{Problème \arabic{#1} continué sur la prochaine page\ldots}\nobreak{}
}

\newcommand{\exitProblemHeader}[1]{
    \nobreak\extramarks{Problème \arabic{#1} (continué)}{Problème \arabic{#1} continué sur la prochaine page\ldots}\nobreak{}
    \stepcounter{#1}
    \nobreak\extramarks{Problème \arabic{#1}}{}\nobreak{}
}

\setcounter{secnumdepth}{0}
\newcounter{partCounter}
\newcounter{homeworkProblemCounter}
\setcounter{homeworkProblemCounter}{1}
\nobreak\extramarks{Problème \arabic{homeworkProblemCounter}}{}\nobreak{}

%
% Homework Problem Environment
%
% This environment takes an optional argument. When given, it will adjust the
% problem counter. This is useful for when the problems given for your
% assignment aren't sequential. See the last 3 problems of this template for an
% example.
%
\newenvironment{homeworkProblem}[1][-1]{
    \ifnum#1>0
        \setcounter{homeworkProblemCounter}{#1}
    \fi
    \section{Problème \arabic{homeworkProblemCounter}}
    \setcounter{partCounter}{1}
    \enterProblemHeader{homeworkProblemCounter}
}{
    \exitProblemHeader{homeworkProblemCounter}
}

%
% Homework Details
%   - Title
%   - Due date
%   - Class
%   - Section/Time
%   - Instructor
%   - Author
%

\newcommand{\hmwkTitle}{Travail\ \#1}
\newcommand{\hmwkDueDate}{21\ mars\ 2022}
\newcommand{\hmwkClass}{Géométrie\ vectorielle}
\newcommand{\hmwkAuthorName}{\textbf{Jean-Philippe\ Miguel-Gagnon}}

%
% Title Page
%

\title{
    \vspace{2in}
    \textmd{\textbf{\hmwkClass:\ \hmwkTitle}}\\
    \normalsize\vspace{0.1in}\small{Due\ le\ \hmwkDueDate}\\
    \vspace{3in}
}

\author{\hmwkAuthorName}
\date{}

\renewcommand{\part}[1]{\textbf{\large Part \Alph{partCounter}}\stepcounter{partCounter}\\}

%
% Various Helper Commands
%

% Useful for algorithms
\newcommand{\alg}[1]{\textsc{\bfseries \footnotesize #1}}

% For derivatives
\newcommand{\deriv}[1]{\frac{\mathrm{d}}{\mathrm{d}x} (#1)}

% For partial derivatives
\newcommand{\pderiv}[2]{\frac{\partial}{\partial #1} (#2)}

% Integral dx
\newcommand{\dx}{\mathrm{d}x}

% Alias for the Solution section header
\newcommand{\solution}{\textbf{\large Solution}}

% Probability commands: Expectation, Variance, Covariance, Bias
\newcommand{\E}{\mathrm{E}}
\newcommand{\Var}{\mathrm{Var}}
\newcommand{\Cov}{\mathrm{Cov}}
\newcommand{\Bias}{\mathrm{Bias}}


\begin{document}

\maketitle

\pagebreak

\begin{homeworkProblem}
  Soit le triangle ABC, de sommets \(A(-2, 3)\), \(B(4, -2)\) et \(C(9, 1)\).

  \[
    \begin{split}
      \vec{BA} &= (-6, 5)
      \\
      \vec{BC} &= (5, 3)
      \\
      \vec{AC} &= (11, -2)
    \end{split}
  \]


  \begin{enumerate}
    \item Trouver la mesure en degrés de l'angle obtus du triangle ABC.
          \[
            \begin{split}
              \theta &= \arccos\frac{\vec{BA}\bullet\vec{BC}}{||\vec{BA}||\cdot||\vec{BC}||}
              \\
              &=\arccos\frac{(-6,5)\bullet(5,3)}{\sqrt{61}\cdot\sqrt{34}}
              \\
              &=\arccos\frac{6(5) + 5(3)}{\sqrt{2074}}
              \\
              &=\arccos\frac{15}{\sqrt{2074}}
              \\
              &\approx\arccos(0.3293)
              \\
              &\approx70.77\deg
              \\
              &\approx109.23\deg
            \end{split}
          \]
    \item Soit \(D\), le point milieu du côté \(\vec{AC}\).
          \begin{enumerate}
            \item Exprimer le vecteur \(\vec{BD}\) comme une combinaison linéaire de vecteurs construits à partir des points \(A\), \(B\), \(C\).
                  \[
                    \begin{split}
                      \vec{BD} &= \vec{BA} + \vec{AD}
                      \\
                      &= \vec{BA} + \frac{1}{2}\vec{AC}
                    \end{split}
                  \]
            \item Trouver la longueur de la médiane \(\overline{BD}\).
                  \[
                    \begin{split}
                      \vec{BD} &= \vec{BA} + \frac{1}{2}\vec{AC}
                      \\
                      &= (-6, 5) + \frac{1}{2}(11, -2)
                      \\
                      &= (-6, 5) + (\frac{11}{2}, -1)
                      \\
                      &= (\frac{1}{2}, 4)
                    \end{split}
                    \begin{split}
                      \overline{BD} &= ||\vec{BA}||
                      \\
                      &= ||(\frac{1}{2}, 4)||
                      \\
                      &= \sqrt{\frac{1}{2}^2 + 4^2}
                      \\
                      &= \frac{\sqrt{65}}{2}
                      \\
                      &\approx 4.03113
                    \end{split}
                  \]
          \end{enumerate}
          \pagebreak
    \item Soit \(h\), la hauteur abaissé du point \(A\) et qui coupe la droite \(\overline{BC}\) à angle droit au point \(E\).
          \begin{enumerate}
            \item Exprimer le vecteur \(\vec{BE}\) à l'aide d'une projection de vecteurs construits à partir des points \(A\), \(B\), \(C\).
                  \[
                    \vec{BE} = \vec{BA}_{\vec{BC}}
                  \]
            \item Trouvez la longueur du segment \(\overline{BE}\).
                  \[
                    \begin{split}
                      \vec{BE} &= \frac{\vec{BA}\bullet\vec{BC}}{||\vec{BC}||}\cdot\vec{BC}
                      \\
                      &=\frac{(-6,5)\bullet(5,3)}{(\sqrt{5^2 + 3^2})^2}\cdot\vec{BC}
                      \\
                      &=\frac{6(5) + 5(3)}{5^2+3^2}\cdot\vec{BC}
                      \\
                      &=\frac{15}{34}\cdot\vec{BC}
                      \\
                      &=\frac{15}{34}\cdot(5, 3)
                      \\
                      &=(\frac{75}{34}, \frac{45}{34})
                      \\
                      &\approx(2.20588, 1.32352)
                    \end{split}
                    \begin{split}
                      \overline{BE} &= ||\vec{BE}||
                    \end{split}
                  \]
          \end{enumerate}
  \end{enumerate}

\end{homeworkProblem}


\end{document}